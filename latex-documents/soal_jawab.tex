\chapter{Latihan Pemantapan Konsep}
\section{Soal}
\begin{enumerate}
    \item Dengan menggunakan NumPy hitunglah jumlah baris dan kolom pada matriks - matriks berikut ini:
        \begin{enumerate}[label=(\Alph*)]
            \item $\begin{bmatrix} 1 & 0 & 0\\0 & 1 & 0\\0 & 0 & 1 \end{bmatrix}$
            \item $\begin{bmatrix}2 & 1 & 1\\ 1 & 2 & 8\end{bmatrix}$
            \item $\begin{bmatrix}1 & -1 & 1\end{bmatrix}$
            \item $\begin{bmatrix}1 & 2 \\ 8 & 1\end{bmatrix}$
            \item $\begin{bmatrix} 1 & 0 \\ 0 & 1\\ 0 & 0\\ 2 & 1 \end{bmatrix}$
            \item $\begin{bmatrix}2 & 1 & 1 & 2\\ 1 & 2 & 2 & 1 \end{bmatrix}$
        \end{enumerate}
    \item Dengan menggunakan NumPy, hitunglah transpos pada matriks - matriks berikut ini:
        \begin{enumerate}[label=(\Alph*)]
            \item $\begin{bmatrix}6 & 5 & 4 & 2\\1 & 5 & 2 & 1\\ 7 & 0 & 2 & 1\\ 1 & 2 & 3 & 4\end{bmatrix}$
            \item $\begin{bmatrix}2 & 1 & 1\\ 1 & 3 & 2\end{bmatrix}$
        \end{enumerate}
    \item Dengan menggunakan NumPy, hitunglah norma $L^2$ dari vektor $\mathbf{x}$ sebagai berikut:
    $\mathbf{x} = \begin{bmatrix}1 & 2 & 3 & 4\end{bmatrix}$
    \item Carilah turunan pertama dari:
    \begin{enumerate}[label=(\Alph*)]
        \item $f(x,y) = x^{4}y + 2x$
        \item $f(x,y) = 2y + 3x^{2}$
    \end{enumerate}
    \item \textbf{Pilihlah jawaban yang paling tepat!}
    
    Arah penurunan dari fungsi objektif $f(x)$ dapat diketahui melalui:
    \begin{enumerate}[label=(\Alph*)]
        \item Gradien $(\nabla)$
        \item Gradien negatif $(-\nabla)$
        \item Matriks Jacobian $(\mathbf{J})$
        \item Laplacian $(\nabla^2)$
    \end{enumerate}
    
    \item \textbf{Lengkapi bagian kosong pada soal sebagai berikut: }
    
    Integral \underline{\hspace{3cm}} dari $f(x)$ merupakan bilangan representasi dari wilayah di bawah kurva dari $x = a$ hingga $x = b$. Integral \underline{\hspace{3cm}} dari $f(x)$ tidak mempunyai batas dan hasil akhirnya berupa suatu fungsi.
    
    \item \textbf{Lengkapi bagian kosong pada soal sebagai berikut: }
    
    Titik kritis dari suatu fungsi konveks sudah pasti merupakan titik minimum \underline{\hspace{3cm}}.
    
    \item Carilah titik minimum global dari fungsi objektif $f(x) = 2x^{2} - 3$!
    \item \textbf{Lengkapi bagian kosong pada soal sebagai berikut: }
    
    \underline{\hspace{3cm}} merupakan sumber dari sifat stokastik yang melekat pada sistem yang dimodelkan, pemodelan yang tidak tepat, dan tidak nya data pengamatan.
    
    \item \textbf{Lengkapi bagian kosong pada soal sebagai berikut: }
    
    Distribusi probabilitas untuk peubah acak kontinyu dinamakan sebagai \underline{\hspace{3cm}}.
    
    \item \textbf{Lengkapi bagian kosong pada soal sebagai berikut: }
    
    Simbol matematis untuk rata - rata, varian, dan standar deviasi adalah: \underline{\hspace{1cm}}, \underline{\hspace{1cm}}, \underline{\hspace{1cm}}.
    
    \item Tuliskanlah persamaan yang digunakan untuk memodelkan ekspektasi pada peubah acak diskrit!
    
    \item \textbf{Lengkapi bagian kosong pada soal sebagai berikut: }
    
    Distribusi probabilitas yang umum dijumpai di alam adalah distribusi \underline{\hspace{3cm}}.
\end{enumerate}

\section{Jawaban}
\begin{enumerate}
\item\\
\lstinputlisting[language=Python]{jwb1.py}
Hasilnya:
\begin{verbatim}
Jumlah baris A: 3, Jumlah kolom A: 3.
Jumlah baris B: 2, Jumlah kolom B: 3.
Jumlah baris C: 3, Jumlah kolom C: 3.
Jumlah baris D: 2, Jumlah kolom D: 2.
Jumlah baris E: 4, Jumlah kolom E: 2.
Jumlah baris F: 2, Jumlah kolom F: 4.
\end{verbatim}

\item\\
\lstinputlisting[language=Python]{jwb2.py}
Hasilnya:
\begin{verbatim}
[[6 1 7 1]
 [5 5 0 2]
 [4 2 2 3]
 [2 1 1 4]]
################
[[2 1]
 [1 3]
 [1 2]]
\end{verbatim}

\item\\
\lstinputlisting[language=Python]{jwb3.py}
Hasilnya:
\begin{verbatim}
Norma Euklidesan dari vektor x adalah: 5.477225575051661
\end{verbatim}

\item 
\begin{enumerate}[label=(\Alph*)]
    \item $\nabla f(x,y) = \left[4x^{3} + 2,\, x^{4}\right]$
    \item $\nabla f(x,y) = \left[6x,\, 2\right]$
\end{enumerate}

\item \textbf{B}. Gradien negatif $(-\nabla)$

\item Integral \textbf{tentu} dari $f(x)$ merupakan bilangan representasi dari wilayah di bawah kurva dari $x = a$ hingga $x = b$. Integral \textbf{tak-tentu} dari $f(x)$ tidak mempunyai batas dan hasil akhirnya berupa suatu fungsi.

\item Titik kritis dari suatu fungsi konveks sudah pasti merupakan titik minimum \textbf{global}. 
\item 
$f(x) = 2x^{3} -3$\\\\
$\frac{df(x)}{dx} = 2(2)x - 4x$\\
\begin{itemize}
\item Turunan pertama dinyatakan dalam bentuk nol:\\
$0 = 4x$\\
$x = 0$ (merupakan titik kritis)
\item Uji coba turunan kedua:\\\\
$\frac{d^{2}f(x)}{dx} = 4$\\\\
Karena turunan kedua bernilai positif dan tidak ada titik - titik kritis lainnya, maka $x=0$ dinyatakan sebagai titik minimum global.
\end{itemize}
\item \textbf{Ketidakpastian} merupakan sumber dari sifat stokastik yang melekat pada sistem yang dimodelkan, pemodelan yang tidak tepat, dan tidak nya data pengamatan.
\item Distribusi probabilitas untuk peubah acak kontinyu dinamakan sebagai \textbf{fungsi kepadatan peluang} (\textbf{probability density function}).
\item Simbol matematis untuk rata - rata, varian, dan standar deviasi adalah: $\mu$, $\sigma^2$, dan $\sigma$.
\item $	E_{x \sim\ p}\left[f(x)\right] = \sum_{x} P(x)f(x)$
\item Distribusi probabilitas yang umum dijumpai di alam adalah distribusi \textbf{Gaussian (normal)}.
\end{enumerate}