\chapter*{Kata Pengantar}

Keilmuan analisis data menjadi primadona dalam perbincangan global di satu dekade terakhir. Hal ini ditandai dengan semakin terintegrasinya kehidupan umat manusia dengan teknologi yang berbuntut dengan lahirnya kumpulan data raksasa (\textit{big data}) yang dapat kemudian dijadikan bahan bagi pemelajaran algoritma kecerdasan buatan untuk menghasilkan produk yang diharapkan dapat beradaptasi dengan kebutuhan pengguna. 

Kecerdasan buatan di dalam beberapa waktu terakhir ini juga menelurkan perkembangan cukup pesat. Hal ini ditandai dengan terciptanya mobil swakemudi (\textit{self-driving car}), diagnosa medis mutakhir, sistem kecerdasan yang dapat mengalahkan manusia dalam permainan catur dan GO, dll.

Progres di dalam penelitian dan pengembangan kecerdasan buatan tampak menjanjikan. Bukan tidak mungkin dalam beberapa dekade mendatang kita dapat membangun sistem kecerdasan buatan umum (\textit{Artificial General Intelligence}) yang dapat menggantikan profesi manusia. Namun untuk mencapai hal tersebut diperlukan kerja keras dan kecerdasan dari para peneliti dan pengembang untuk menemukan dan mengeksekusi algoritma kecerdasan buatan hingga mencapai tahap siap untuk diproduksi secara massal. Mimpi ini tentu tidaklah mudah untuk dicapai karena untuk berkontribusi di dalam penelitian dan pengembangan algoritma kecerdasan buatan diperlukan dasar kemampuan matematis dan pemrograman yang cukup kuat.

Melalui catatan singkat ini penyusun hendak mengajak pembaca untuk berkenalan dengan konsep - konsep matematika yang diperlukan sebagai fondasi untuk memahami dan mengimplementasikan algoritma - algoritma yang digunakan di dalam analisis data melalui bahasa pemrograman Python dan R (penyusun mengharapkan familiaritas pembaca terhadap konsep - konsep kedua bahasa pemrograman tersebut sebelum mempelajari buku ini). 

Catatan ini sendiri didesain sebagai jembatan kesenjangan pengetahuan di dunia industri dan dunia akademik terkait konsep - konsep dasar matematika yang dibutuhkan bagi para analis data junior, pengembang piranti lunak, mahasiswi/a dari jurusan non-eksakta, dan kaum awam guna berkontribusi di dalam pengembangan kecerdasan buatan melalui analisis data.

Buku ini akan membahas tiga konsep matematika utama yang menjadi fondasi algoritma kecerdasan buatan jaringan saraf tiruan (\textit{neural networks}), yakni aljabar linier, kalkulus peubah banyak, dan teori probabilitas. 

Selain konsep - konsep matematika, pembaca juga akan disuguhkan laboratorium pemrograman kecil pada setiap bab-nya dan latihan pemahaman konsep di bagian akhir buku. Pembaca diharapkan dapat memahami dan mengimplementasikan konsep - konsep matematika yang diterangkan di sini pada bidang keilmuannya masing - masing ketika selesai mempelajari catatan ini. 

Catatan ini juga bersifat sumber terbuka, karena dituliskan dengan menggunakan \LaTeX{} dan berlisensi milik publik (\textit{copy-left}), sehingga para pembaca dapat menyalin dan mengubahnya, bahkan untuk kepentingan komersil sekalipun, secara cuma - cuma di \url{https://github.com/sandyherho/buku-mat-andat}. 

Akhir kata, semoga catatan singkat dapat membantu dan saya menanti dengan tangan terbuka kolaborasi pembelajaran berbasis kode terbuka di laman GitHub penyusun.  

\mbox{}\\
%\mbox{}\\
\noindent Penyusun \\\\
